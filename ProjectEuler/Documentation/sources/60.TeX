\prob{51}
By replacing the $1^{\text{st}}$ digit of *3, it turns out that six of the nine possible values: 13, 23, 43, 53, 73, and 83, are all prime.

By replacing the 3$^\text{rd}$ and 4$^\text{th}$ digits of 56**3 with the same digit, this 5-digit number is the first example having seven primes among the ten generated numbers, yielding the family: 56003, 56113, 56333, 56443, 56663, 56773, and 56993. Consequently 56003, being the first member of this family, is the smallest prime with this property.

Find the smallest prime which, by replacing part of the number (not necessarily adjacent digits) with the same digit, is part of an eight prime value family.

\sol{51}

Let $\{p, p+d, p+2d, \dots, p+9d\}$ be the ten numbers, with $d$ consisting of ones and zeros.  Then if 8 of those prime,
then we must have $3|d$, or at least one of the numbers will be divisible by 3.  Hence $d$ must consist of $3k$ 1's, with some zeros.  We look for a solution with $k = 1$.  Also, note that the last digit of primes is one of $\{1, 3, 7, 9\}$, the last digit has to be a fixed one.

Now create a set of possible 3-digit positions of the ones in $d$.  For numbers up to 6 digits long, there are $\binom 5 3 = 10$ such maps, so store them in the format $[d, 10^a, 10^b, 10^c]$, where the last three are the multipliers for the other digits.  Now run through the list of all possible sets of 3 digits (so 1-1000), trying out how many of the numbers, with the map inserted will make primes, by accessing a boolean prime table.  The answer pops out.

\ans{121313}


%%%%%%%%%%%%%%%%%%%%%%%%%%%%%%%%%%%%%%%%%%%%%%%%%%%%%%%%%%%%%%%%%%%%%%%%%%%%%%%%

\prob{52}
It can be seen that the number, 125874, and its double, 251748, contain exactly the same digits, but in a different order.

Find the smallest positive integer, $x$, such that $2x$, $3x$, $4x$, $5x$, and $6x$, contain the same digits.

\sol{52}
As in problem 49, write a function that checks whether two numbers consist of the same digits.  As the six numbers will have different leading digits, we start at $N = 123456$, and check whether multiples of it have the same digits.  The answer comes
out to be the recurring digits in the expansion of 1/7.

\ans{142857}


%%%%%%%%%%%%%%%%%%%%%%%%%%%%%%%%%%%%%%%%%%%%%%%%%%%%%%%%%%%%%%%%%%%%%%%%%%%%%%%%

\prob{53}
There are exactly ten ways of selecting three from five, 12345:
$$ 123, 124, 125, 134, 135, 145, 234, 235, 245, \hbox{ and } 345.$$
In combinatorics, we use the notation, $^5C_3 = 10$.

In general, $^nC_r = \frac{n!}{r!(n-r)!}$, where
$r \leq n$, $n! = n\times(n-1)\times\dots\times3\times2\times1$, and 0! = 1.

It is not until $n = 23$, that a value exceeds one-million: $^{23}C_{10} = 1144066$.

How many, not necessarily distinct, values of  $^nC_r$, for $1 \leq n \leq 100$, are greater than one-million?

\sol{53}
Ok, fairly straight forward.  Write a function to evaluate $C(n,r)$, and run through the 5000-odd possible inputs,
evaluating each one.

\ans{4075}


%%%%%%%%%%%%%%%%%%%%%%%%%%%%%%%%%%%%%%%%%%%%%%%%%%%%%%%%%%%%%%%%%%%%%%%%%%%%%%%%

\prob{54}
In the card game poker, a hand consists of five cards and are ranked, from lowest to highest, in the following way:
\begin{itemize}
\vspace{-0.25cm}    \item High Card: Highest value card.
\vspace{-0.25cm}    \item One Pair: Two cards of the same value.
\vspace{-0.25cm}    \item Two Pairs: Two different pairs.
\vspace{-0.25cm}    \item Three of a Kind: Three cards of the same value.
\vspace{-0.25cm}    \item Straight: All cards are consecutive values.
\vspace{-0.25cm}    \item Flush: All cards of the same suit.
\vspace{-0.25cm}    \item Full House: Three of a kind and a pair.
\vspace{-0.25cm}    \item Four of a Kind: Four cards of the same value.
\vspace{-0.25cm}    \item Straight Flush: All cards are consecutive values of same suit.
\vspace{-0.25cm}    \item Royal Flush: Ten, Jack, Queen, King, Ace, in same suit.
\end{itemize}
\vspace{-0.25cm}
The cards are valued in the order:
2, 3, 4, 5, 6, 7, 8, 9, 10, J, Q, K, A.

If two players have the same ranked hands then the rank made up of the highest value wins; for example, a pair of eights beats a pair of fives (see example 1 below). But if two ranks tie, for example, both players have a pair of queens, then highest cards in each hand are compared; if the highest cards tie then the next highest cards are compared, and so on.

The file, \verb"poker.txt", contains one-thousand random hands dealt to two players. Each line of the file contains ten cards (separated by a single space): the first five are Player 1's cards and the last five are Player 2's cards. You can assume that all hands are valid (no invalid characters or repeated cards), each player's hand is in no specific order, and in each hand there is a clear winner.

How many hands does Player 1 win?

\vspace{-0.25cm}
\sol{54}
Meh, messy but boring.  Check each hand for flush, straight, full house and trips -- there are no ties in between these.
For pairs and nothings, there are ties, but all can be resolved by highest pair, or highest card, so code does not even
have to be that rigorous.
\vspace{-0.25cm}

\ans{376}


%%%%%%%%%%%%%%%%%%%%%%%%%%%%%%%%%%%%%%%%%%%%%%%%%%%%%%%%%%%%%%%%%%%%%%%%%%%%%%%%

\prob{55}
If we take 47, reverse and add, 47 + 74 = 121, which is palindromic.

Not all numbers produce palindromes so quickly. For example,
\begin{eqnarray*}
349 + 943 &= &1292 \\
1292 + 2921 &= &4213 \\
4213 + 3124 &= &7337
\end{eqnarray*}
That is, 349 took three iterations to arrive at a palindrome.

Although no one has proved it yet, it is thought that some numbers, like 196, never produce a palindrome. A number that never forms a palindrome through the reverse and add process is called a Lychrel number. Due to the theoretical nature of these numbers, and for the purpose of this problem, we shall assume that a number is Lychrel until proven otherwise. In addition you are given that for every number below ten-thousand, it will either (i) become a palindrome in less than fifty iterations, or, (ii) no one, with all the computing power that exists, has managed so far to map it to a palindrome. In fact, 10677 is the first number to be shown to require over fifty iterations before producing a palindrome: 4668731596684224866951378664 (53 iterations, 28-digits).

Surprisingly, there are palindromic numbers that are themselves Lychrel numbers; the first example is 4994.

How many Lychrel numbers are there below ten-thousand?

\footnotesize
NOTE: Wording was modified slightly on 24 April 2007 to emphasise the theoretical nature of Lychrel numbers.

\normalsize
\sol{55}
The trick is to write a function which reverses an integer.  The palindrome check is then \verb"Rev(n) == n",
otherwise increment $n$ by it's reversal.  If done 50 times, stop and return an affirmative.

\ans{249}


%%%%%%%%%%%%%%%%%%%%%%%%%%%%%%%%%%%%%%%%%%%%%%%%%%%%%%%%%%%%%%%%%%%%%%%%%%%%%%%%

\prob{56}
A googol ($10^{100})$ is a massive number: one followed by one-hundred zeros; $100^{100}$ is almost unimaginably large: one followed by two-hundred zeros. Despite their size, the sum of the digits in each number is only 1.

Considering natural numbers of the form, $a^{b}$, where $a, b < 100$, what is the maximum digital sum?

\sol{56}
A little uninspirational, one can write a function that evaluates the digit sum, then run through the $100^2$ possible
numbers and look for the maximum.  It turns out to be for $99^{95}$.

\ans{972}


%%%%%%%%%%%%%%%%%%%%%%%%%%%%%%%%%%%%%%%%%%%%%%%%%%%%%%%%%%%%%%%%%%%%%%%%%%%%%%%%

\prob{57}
It is possible to show that the square root of two can be expressed as an infinite continued fraction.
$$\sqrt2 = 1 + 1/(2 + 1/(2 + 1/(2 + \dots ))) = 1.414213...$$
By expanding this for the first four iterations, we get:
\begin{eqnarray*}
&&1 + 1/2     = 3/2 = 1.5\\
&&1 + 1/(2 + 1/2) = 7/5 = 1.4 \\
&&1 + 1/(2 + 1/(2 + 1/2)) = 17/12 = 1.41666... \\
&&1 + 1/(2 + 1/(2 + 1/(2 + 1/2))) = 41/29 = 1.41379...
\end{eqnarray*}
The next three expansions are 99/70, 239/169, and 577/408, but the eighth expansion, 1393/985, is the first example where the number of digits in the numerator exceeds the number of digits in the denominator.

In the first one-thousand expansions, how many fractions contain a numerator with more digits than denominator?

\sol{57}
The expression for the numerator and denominator can be written recursively, as
$N_{i+1} = 2*N_i+D_i$ and $D_{i+1} = N_i+D_i$.  By introducing the highest power of 10 lower than
$N$, we can check if that is bigger than $D$ in each case, and summing those gives the answer.

\ans{153}


%%%%%%%%%%%%%%%%%%%%%%%%%%%%%%%%%%%%%%%%%%%%%%%%%%%%%%%%%%%%%%%%%%%%%%%%%%%%%%%%

\prob{58}
Starting with 1 and spiralling anticlockwise in the following way, a square spiral with side length 7 is formed.
\begin{center}
\begin{figure}[h]
\centering
\includegraphics[width = 0.40\textwidth]{./images/p_058.png}
\end{figure}
\end{center}
\vspace{-1cm}
It is interesting to note that the odd squares lie along the bottom right diagonal, but what is more interesting is that 8 out of the 13 numbers lying along both diagonals are prime; that is, a ratio of $8/13 \approx 62\%$.

If one complete new layer is wrapped around the spiral above, a square spiral with side length 9 will be formed. If this process is continued, what is the side length of the square spiral for which the ratio of primes along both diagonals first falls below 10\%?

\sol{58}
As usual we need to generate a list of primes.  Then create the spiral, checking whether the entries on the three diagonals are prime (the fourth contains perfect squares).  The numbers get a bit large for a boolean array, so have to check them prime-by-prime, making this a fairly slow procedure.

\ans{26241}


%%%%%%%%%%%%%%%%%%%%%%%%%%%%%%%%%%%%%%%%%%%%%%%%%%%%%%%%%%%%%%%%%%%%%%%%%%%%%%%%

\prob{59}
Each character on a computer is assigned a unique code and the preferred standard is ASCII (American Standard Code for Information Interchange). For example, uppercase A $ = 65$, asterisk * $ = 42$, and lowercase k $ = 107$.

A modern encryption method is to take a text file, convert the bytes to ASCII, then XOR each byte with a given value, taken from a secret key. The advantage with the XOR function is that using the same encryption key on the cipher text, restores the plain text; for example, 65 XOR 42 = 107, then 107 XOR 42 = 65.

For unbreakable encryption, the key is the same length as the plain text message, and the key is made up of random bytes. The user would keep the encrypted message and the encryption key in different locations, and without both ``halves'', it is impossible to decrypt the message.

Unfortunately, this method is impractical for most users, so the modified method is to use a password as a key. If the password is shorter than the message, which is likely, the key is repeated cyclically throughout the message. The balance for this method is using a sufficiently long password key for security, but short enough to be memorable.

Your task has been made easy, as the encryption key consists of three lower case characters. Using \verb"cipher1.txt", a file containing the encrypted ASCII codes, and the knowledge that the plain text must contain common English words, decrypt the message and find the sum of the ASCII values in the original text.

\sol{59}
The key is to notice the most common character in any text message is the space (32).  So by checking for the commonest occurrences, can guess what the three characters are.  There are six orderings of those, and humans are so much better at work recognition that programs.  Nonetheless, can calculate number of spaces -- the one with the most is the correct one.  Can reasonably expect every sixth character to be a space.


\ans{107359}


%%%%%%%%%%%%%%%%%%%%%%%%%%%%%%%%%%%%%%%%%%%%%%%%%%%%%%%%%%%%%%%%%%%%%%%%%%%%%%%%

\prob{60}
The primes 3, 7, 109, and 673, are quite remarkable. By taking any two primes and concatenating them in any order the result will always be prime. For example, taking 7 and 109, both 7109 and 1097 are prime. The sum of these four primes, 792, represents the lowest sum for a set of four primes with this property.

Find the lowest sum for a set of five primes for which any two primes concatenate to produce another prime.

\sol{60}
This one is messy.  Need to find a 5-tuple of primes with this property.  The fairly crude implementation works like this:\\
Consider all primes up to 10000 (as a start).  Then for each pair of primes, can make a boolean array whether they are prime or not.  This can be placed inside a $1230\times1230$ table, though filling it requires checking a lot of 8-digit numbers from primeness, so is slow.  Once that is done, run 5-variable loop, with largest on the outside, looking for 5 primes that satisfy the condition (with checking within each and every loop).  This still runs in under 30s on the laptop.  
Solution set is $\{8389, 6733, 5701, 5197, 13\}$.  

\footnotesize
Note: code does not show that this is the minimum -- the first solution found checks out.

\normalsize
\ans{26033}

