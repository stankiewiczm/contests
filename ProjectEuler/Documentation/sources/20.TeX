\prob{11}
In the $20\times20$ grid below, four numbers along a diagonal line have been marked in red.
\begin{center}
\begin{figure}[h]
\centering
\includegraphics[width = 0.8\textwidth]{./images/p_011.png}
\end{figure}
\end{center}
\vspace{-1cm}
The product of these numbers is $26 \times 63 \times 78 \times 14 = 1788696.$

What is the greatest product of four adjacent numbers in any direction (up, down, left, right, or diagonally) in the $20\times20$ grid?

\sol{11}
Nothing special here.  Just evaluate every product of 4 numbers in each direction (down, right, two diagonals), being careful
not to overshoot the grid.  Best set is (87, 97, 94, 89) along a upward diagonal.

\ans{70600674}


%%%%%%%%%%%%%%%%%%%%%%%%%%%%%%%%%%%%%%%%%%%%%%%%%%%%%%%%%%%%%%%%%%%%%%%%%%%%%%%%

\prob{12}
The sequence of triangle numbers is generated by adding the natural numbers. So the $7^{th}$ triangle number would be
$1 + 2 + 3 + 4 + 5 + 6 + 7 = 28$. The first ten terms would be:
$$1, 3, 6, 10, 15, 21, 28, 36, 45, 55, ...$$
Let us list the factors of the first seven triangle numbers:
\begin{eqnarray*}
     1&: &1        \\
     3&: &1,3      \\
     6&: &1,2,3,6  \\
    10&: &1,2,5,10 \\
    15&: &1,3,5,15 \\
    21&: &1,3,7,21 \\
    28&: &1,2,4,7,14,28
\end{eqnarray*}
We can see that 28 is the first triangle number to have over five divisors.

What is the value of the first triangle number to have over five hundred divisors?

\sol{12}
Notice that a triangular number is $T_n = \frac{n(n+1)}2$, and if we look at the prime expansion of a number, the number
of factors is easily found:
$$n = \prod_{i=1}^k p_i^{\alpha_i} \qquad \Rightarrow \qquad d(n) = \prod_{i=1}^k (\alpha_i+1).$$
As $n$ and $n+1$ are relatively prime, we know that
$$d\left(\frac{n(n+1)}2\right) = d\left( \frac n2 \right) \times d(n+1) \qquad \hbox{ for even }n,$$
and similarly for odd $n$.  So we loop over the $n$, find the factors of the two terms and check if the product exceeds 500.
This happens first for the $T_{12375}$.


\ans{76576500}


%%%%%%%%%%%%%%%%%%%%%%%%%%%%%%%%%%%%%%%%%%%%%%%%%%%%%%%%%%%%%%%%%%%%%%%%%%%%%%%%

\prob{13}
Work out the first ten digits of the sum of the following one-hundred 50-digit numbers.
\begin{center}
\verb"37107287533902102798797998220837590246510135740250"
\verb"46376937677490009712648124896970078050417018260538"
\verb"74324986199524741059474233309513058123726617309629" \\
$\vdots$\\
\verb"72107838435069186155435662884062257473692284509516"
\verb"20849603980134001723930671666823555245252804609722"
\verb"53503534226472524250874054075591789781264330331690"
\end{center}

\sol{13}
Just paste the whole lot of numbers into a list, and find the sum.  Divide by $10^{42}$ to keep the first 10 digits.


\ans{5537376230}


%%%%%%%%%%%%%%%%%%%%%%%%%%%%%%%%%%%%%%%%%%%%%%%%%%%%%%%%%%%%%%%%%%%%%%%%%%%%%%%%

\prob{14}
The following iterative sequence is defined for the set of positive integers:
\begin{eqnarray*}
n &\rightarrow &n/2 \;\;  (n \hbox{ is even)} \\
n &\rightarrow &3n + 1 \; (n \hbox{ is odd)}
\end{eqnarray*}
Using the rule above and starting with 13, we generate the following sequence:
$$13 \rightarrow 40 \rightarrow 20 \rightarrow 10 \rightarrow 5 \rightarrow 16 \rightarrow 8 \rightarrow 4 \rightarrow 2 \rightarrow 1.$$
It can be seen that this sequence (starting at 13 and finishing at 1) contains 10 terms. Although it has not been proved yet (Collatz Problem), it is thought that all starting numbers finish at 1.

Which starting number, under one million, produces the longest chain?

\sol{14}
Begin at 1, and evaluate the length of the Collatz sequence for each $n$.  For each $n$, if after $k$ steps
we reach a number $m$ less than $n$, there is no need to iterate further as we know how many steps it takes to get from
there: \verb"Time[n] = Time[m]+k".

\ans{871}


%%%%%%%%%%%%%%%%%%%%%%%%%%%%%%%%%%%%%%%%%%%%%%%%%%%%%%%%%%%%%%%%%%%%%%%%%%%%%%%%

\prob{15}
Starting in the top left corner of a $2\times2$ grid, there are 6 routes (without backtracking) to the bottom right corner.
\begin{center}
\begin{figure}[h]
\centering
\includegraphics{./images/p_015.png}
\end{figure}
\end{center}
\vspace{-1cm}
How many routes are there through a $20\times20$ grid?

\sol{15}
On a $n\times n$ grid, we make a total of $2n$ steps, of which $n$ are down and $n$ are across.
The number of paths corresponds to the number of ways of choosing the $n$ down steps out of the $2n$ total,
so $\binom{2n}{n}$.

\ans{137846528820}


%%%%%%%%%%%%%%%%%%%%%%%%%%%%%%%%%%%%%%%%%%%%%%%%%%%%%%%%%%%%%%%%%%%%%%%%%%%%%%%%
\prob{16}
$2^{15} = 32768$ and the sum of its digits is $3 + 2 + 7 + 6 + 8 = 26$.

What is the sum of the digits of the number $2^{1000}$?

\sol{16}
The \verb"Python" long integers come in useful here.  We simply evaluate $2^{1000}$,
and then keep dividing the number by 10, summing the remainders until we reach zero.

\ans{1366}


%%%%%%%%%%%%%%%%%%%%%%%%%%%%%%%%%%%%%%%%%%%%%%%%%%%%%%%%%%%%%%%%%%%%%%%%%%%%%%%%

\prob{17}
If the numbers 1 to 5 are written out in words: one, two, three, four, five, then there are 3 + 3 + 5 + 4 + 4 = 19 letters used in total.

If all the numbers from 1 to 1000 (one thousand) inclusive were written out in words, how many letters would be used?

\footnotesize
NOTE: Do not count spaces or hyphens. For example, 342 (three hundred and forty-two) contains 23 letters and 115 (one hundred and fifteen) contains 20 letters. The use of ``and'' when writing out numbers is in compliance with British usage.
\normalsize

\sol{17} This is a bit of a slog.  The number of letters in the numbers 1 to 19 is hardcoded, as well as from 20 to 90.
The letters used for other numbers less than 100 are a sum of the letters in the ``twenty'' plus the letters in the digit.
Numbers of the form $100k$ have seven digits more than $k$, and numbers of the form $100k+l$ have three more letters
than $100k$ and $l$ individually (allowing for the ``and'').  Finally, 1000 has eleven letters.  Summing up gives the answer.

\ans{21124}


%%%%%%%%%%%%%%%%%%%%%%%%%%%%%%%%%%%%%%%%%%%%%%%%%%%%%%%%%%%%%%%%%%%%%%%%%%%%%%%%

\prob{18}
By starting at the top of the triangle below and moving to adjacent numbers on the row below, the maximum total from top to bottom is 23.
\begin{center}
\begin{figure}[h]
\centering
\includegraphics[width = 0.12\textwidth]{./images/p_018.png}
\end{figure}
\end{center}
\vspace{-1cm}
That is, $3 + 7 + 4 + 9 = 23$.

Find the maximum total from top to bottom of the triangle below:
\begin{center}
\begin{figure}[h]
\centering
\includegraphics[width = 0.6\textwidth]{./images/p_018-2.png}
\end{figure}
\end{center}
\vspace{-1cm}

\footnotesize
NOTE: As there are only 16384 routes, it is possible to solve this problem by trying every route. However, Problem 67, is the same challenge with a triangle containing one-hundred rows; it cannot be solved by brute force, and requires a clever method! ;o)

\normalsize

\sol{18}
This is the first problem to use dynamic programming (DP).  Make a triangular array $A$ of data, and a blank one of $B$.
Set the top point of $B$ to be the top point of $A$.  Now we let $B$ store the highest possible total of a path to end at
each point.  The triangle edges have no choice, but for the middle parts we choose the higher of the two points above the one
we want, and add the value at $A$ to it.  We fill $B$ out row by row, and find the maximum of the bottom row for the answer.

\ans{1074}


%%%%%%%%%%%%%%%%%%%%%%%%%%%%%%%%%%%%%%%%%%%%%%%%%%%%%%%%%%%%%%%%%%%%%%%%%%%%%%%%

\prob{19}
You are given the following information, but you may prefer to do some research for yourself.
\begin{itemize}
\item 1 Jan 1900 was a Monday.
\item Thirty days has September,\\
      April, June and November.\\
      All the rest have thirty-one,\\
      Saving February alone,\\
      Which has twenty-eight, rain or shine.\\
      And on leap years, twenty-nine.
\item A leap year occurs on any year evenly divisible by 4, but not on a century unless it is divisible by 400.
\end{itemize}
How many Sundays fell on the first of the month during the twentieth century (1 Jan 1901 to 31 Dec 2000)?

\sol{19}
The year 1900 had 365 days, so 1 Jan 1901 was a Tuesday.  Then we just need to iterate through the 1200 months
in the 20$^{th}$ century, finding the day on which each started, and reading off the Sundays for the answer.

\ans{171}


%%%%%%%%%%%%%%%%%%%%%%%%%%%%%%%%%%%%%%%%%%%%%%%%%%%%%%%%%%%%%%%%%%%%%%%%%%%%%%%%

\prob{20}
$n!\hbox{ means }n \times (n - 1) \times ... \times 3 \times 2 \times 1.$

Find the sum of the digits in the number 100!

\sol{20}
Like in problem 16, simply evaluate 100!, and sum the digits.

\ans{648}
