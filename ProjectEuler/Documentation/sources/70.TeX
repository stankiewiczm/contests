\prob{61}
Triangle, square, pentagonal, hexagonal, heptagonal, and octagonal numbers are all figurate (polygonal) numbers and are generated by the following formulae:

\begin{tabular}{lll}
{Triangle  } & $ P_{3,n}=n(n+1)/2 	$ &	 1, 3, 6, 10, 15, ...  \\
{Square }    & $ P_{4,n}=n^(2) 	  	$ &  1, 4, 9, 16, 25, ...  \\
{Pentagonal} & $ P_{5,n}=n(3n-1)/2 	$ &  1, 5, 12, 22, 35, ... \\
{Hexagonal } & $ P_{6,n}=n(2n-1) 	$ &  1, 6, 15, 28, 45, ... \\
{Heptagonal} & $ P_{7,n}=n(5n-3)/2 	$ &  1, 7, 18, 34, 55, ... \\
{Octagonal}  & $ P_{8,n}=n(3n-2) 	$ &  1, 8, 21, 40, 65, ...
\end{tabular}

The ordered set of three 4-digit numbers: 8128, 2882, 8281, has three interesting properties.
\begin{enumerate}
   \item The set is cyclic, in that the last two digits of each number is the first two digits of the next number (including the last number with the first).
   \item Each polygonal type: triangle ($P_{3,127}=8128$), square ($P_{4,91}=8281$), and pentagonal ($P_{5,44}=2882$), is represented by a different number in the set.
   \item This is the only set of 4-digit numbers with this property.
\end{enumerate}
Find the sum of the only ordered set of six cyclic 4-digit numbers for which each polygonal type: triangle, square, pentagonal, hexagonal, heptagonal, and octagonal, is represented by a different number in the set.

\sol{61}
Well step one is to generate all these numbers -- there are between 68 and 40 of each.  Thankfully they generally don't match up front-to-end, so if we choose an ordering, the 6-times nested loop evaluates very quickly.  Then we must work through each of the 120 possible orderings -- the very dirty way (5-deep nested loop from 1 to 5) will work fine, as it only performs 3000 excess entries.  In the end we go: $(8256\to5625\to2512\to1281\to8128\to2882)$, which is
Triangle$\to$Square$\to$Heptagon$\to$Octagon$\to$Hexagon$\to$Pentagon$\to$Triangle.


\ans{28684}


%%%%%%%%%%%%%%%%%%%%%%%%%%%%%%%%%%%%%%%%%%%%%%%%%%%%%%%%%%%%%%%%%%%%%%%%%%%%%%%%

\prob{62}
The cube, 41063625 ($345^3$), can be permuted to produce two other cubes: 56623104 ($384^3$) and 66430125 ($405^3$). In fact, 41063625 is the smallest cube which has exactly three permutations of its digits which are also cube.

Find the smallest cube for which exactly five permutations of its digits are cube.

\sol{62}
The program can be run for each number of digits.  Starting at 8-digit cubes, sort all the digits in each cube of length $n$, add to a big list, sort, and check for 5 consecutive entries.  Once the required repeat is found, run through the cubes again and find the smallest one that gets sorted into the required form.

\ans{127035954683}


%%%%%%%%%%%%%%%%%%%%%%%%%%%%%%%%%%%%%%%%%%%%%%%%%%%%%%%%%%%%%%%%%%%%%%%%%%%%%%%%

\prob{63}
The 5-digit number, $16807=7^5$, is also a fifth power. Similarly, the 9-digit number, $134217728=8^9$, is a ninth power.

How many $n$-digit positive integers exist which are also an $n^\text{th}$ power?

\sol{63}
Observe that if $a \geq 10$, then $a^n$ will have at least $n+1$ digits. So need only consider bases of at most 9.
Then, if $a^n$ has less than $n$ digits, all higher powers will also have too few digits.  A simple double loop does the rest.

\ans{49}


%%%%%%%%%%%%%%%%%%%%%%%%%%%%%%%%%%%%%%%%%%%%%%%%%%%%%%%%%%%%%%%%%%%%%%%%%%%%%%%%

\prob{64}
All square roots are periodic when written as continued fractions and can be written in the form:
$\sqrt{n} = a_0 + \frac{1}{a_1+\frac{1}{a_2+\cdot}}$
For example, let us consider $\sqrt{23}$:
$$\sqrt{23} = 4 + \sqrt{23} - 4 = 4 + \frac{1}{\frac{1}{\sqrt{23}-4}} = 4 + \frac{1}{1+\frac{\sqrt{23}-3}7}$$
If we continue we would get the following expansion:
$\sqrt{23} = 4 + \frac{1}{1+\frac{1}{3+\frac{1}{4+\cdots}}}.$
The process can be summarised as follows:
\begin{eqnarray*}
\frac{1}{\sqrt{23}-3} = 1+\frac{\sqrt{23}-3}7 &\qquad &\frac{7}{\sqrt{23}-3} = 3+\frac{\sqrt{23}-3}2 \\
\frac{2}{\sqrt{23}-3} = 1+\frac{\sqrt{23}-4}7 &\qquad &\frac{7}{\sqrt{23}-4} = 8 + \sqrt{23}-4,
\end{eqnarray*}
after which it can be seen that the sequence is repeating. For conciseness, we use the notation $\sqrt{23} = [4;(1,3,1,8)]$, to indicate that the block $(1,3,1,8)$ repeats indefinitely.

The first ten continued fraction representations of irrational square roots are:

\vspace{-0.8cm}
\begin{center}
\begin{tabular}{llp{0.75cm}ll}
$\sqrt2=[1;(2)]$,       & period=1 &&
$\sqrt3=[1;(1,2)]$,     & period=2 \\
$\sqrt5=[2;(4)]$,       & period=1 &&
$\sqrt6=[2;(2,4)]$,     & period=2 \\
$\sqrt7=[2;(1,1,1,4)]$, & period=4 &&
$\sqrt8=[2;(1,4)]$,     & period=2 \\
$\sqrt{10}=[3;(6)]$,    & period=1 &&
$\sqrt{11}=[3;(3,6)]$,  & period=2 \\
$\sqrt{12}= [3;(2,6)]$, & period=2 &&
$\sqrt{13}=[3;(1,1,1,1,6)]$, & period=5 \\
\end{tabular}
\end{center}
\vspace{-0.3cm}

Exactly four continued fractions, for $N \leq 13$, have an odd period.\\
How many continued fractions for $N \leq 10000$ have an odd period?

\vspace{-0.3cm}
\sol{64}
It is clear that the first number, $\lfloor \sqrt{n} \rfloor$ is not generally part of the recurring section.  So start off
with $\sqrt{n} - \lfloor \sqrt{n} \rfloor$.  This can be written as $k + \frac{\sqrt{n}-N}D$.  So in the above example, we have $[4,1,4]$.  Now a little algebra will show that the next values are $k' \equiv \lfloor D/(\sqrt{n}-N)\rfloor$, $D' \equiv (n - N^2)/D$ and $N' \equiv k'D'-N$.  So iterating it we can repeat until the triplet $[k,N,D]$ has been before, which means that the sequence has started recurring.  Then it is just a question of checking the length of the list.

\vspace{-0.3cm}
\ans{1322}


%%%%%%%%%%%%%%%%%%%%%%%%%%%%%%%%%%%%%%%%%%%%%%%%%%%%%%%%%%%%%%%%%%%%%%%%%%%%%%%%

\prob{65}
The square root of 2 can be written as an infinite continued fraction.
$\sqrt 2 = 1 + 	\frac{1}{2+\frac{1}{2+\frac{1}{2+\cdots}}}$
The infinite continued fraction can be written, $\sqrt2 = [1;(2)]$, (2) indicates that 2 repeats ad infinitum. In a similar way, $\sqrt{23} = [4;(1,3,1,8)]$.

It turns out that the sequence of partial values of continued fractions for square roots provide the best rational approximations. Let us consider the convergents for $\sqrt{2}$.
$$ 1+\frac{1}{2} = \frac 32 \qquad 1+\frac{1}{2+\frac 12} = \frac 75 \qquad \frac{1}{2+\frac{1}{2+\frac{1}2}} = \frac{17}{12} \qquad \frac{1}{2+\frac{1}{2+\frac{1}{2+\frac 12}}} = \frac{41}{29} $$
Hence the sequence of the first ten convergents for $\sqrt{2}$ are:
$$1,\; 3/2,\; 7/5,\; 17/12,\; 41/29,\; 99/70,\; 239/169,\; 577/408,\; 1393/985,\; 3363/2378$$
What is most surprising is that the important mathematical constant,
$$e = [2; 1,2,1, 1,4,1, 1,6,1 , \dots , 1,2k,1, \dots].$$
The first ten terms in the sequence of convergents for $e$ are:
$$2, \;3, \;8/3, \;11/4, \;19/7, \;87/32, \;106/39, \;193/71, \;1264/465, \;1457/536, ...$$
The sum of digits in the numerator of the $10^{\text{th}}$ convergent is $1+4+5+7=17$.

Find the sum of digits in the numerator of the $100^{\text{th}}$ convergent of the continued fraction for $e$.

\sol{65}
This is a question of starting at the end (or bottom) of the continuing fraction.  The form of the continued fraction is given, so we start off with $N_1 = L[100]$, and $D_1 = 1$.  Then we work backwards, as follows:
$$N_{i+1} = L[100-i-1]\cdot N_i + D_i \qquad \hbox{and } \qquad D_{i+1} = N_i.$$
With the Python long integers, there is no worry about integer overlows.  It turns out that the 100$^\text{th}$ convergent fraction for $e$ is
$$\frac{N_{100}}{D_{100}} = \frac{6963524437876961749120273824619538346438023188214475670667}{2561737478789858711161539537921323010415623148113041714756}.$$

\vspace{-0.5cm}
\ans{272}


%%%%%%%%%%%%%%%%%%%%%%%%%%%%%%%%%%%%%%%%%%%%%%%%%%%%%%%%%%%%%%%%%%%%%%%%%%%%%%%%

\prob{66}
Consider quadratic Diophantine equations of the form:
$$ x^2 � Dy^2 = 1$$
For example, when $D=13$, the minimal solution in $x$ is $649^2 � 13\times180^2 = 1.$
It can be assumed that there are no solutions in positive integers when $D$ is square.
By finding minimal solutions in $x$ for $D = \{2, 3, 5, 6, 7\}$, we obtain the following:
\begin{eqnarray*}
3^2 - 2\times2^2 &= &1 \\
2^2 - 3\times1^2 &= &1 \\
9^2 - 5\times4^2 &= &1 \\
5^2 - 6\times2^2 &= &1 \\
8^2 - 7\times3^2 &= &1
\end{eqnarray*}
Hence, by considering minimal solutions in $x$ for $D \leq 7$, the largest $x$ is obtained when $D=5$.

Find the value of $D \leq 1000$ in minimal solutions of $x$ for which the largest value of $x$ is obtained.

\vspace{-0.5cm}
\sol{66}
Well this question is a bit of beast, but with a quick solution, if you know how.  The best hint is that it comes directly after questions 64 and 65.  Those questions involved finding the continued fraction expressions of irrational square roots, and the evaluation of a finite sequence of them.  It was noted that these are also the best rational approximations to the roots.

Now if $x^2 - Dy^2 = 1$, then $\tfrac xy \approx \sqrt{D}$.  So the idea is to take `good approximations' to $\sqrt{D}$ and see if they satisfy this equation (also known as Pell's).  Adapting the code from the previous two questions is then easy.  For each $D$, find the continued fraction expansion, at each step evaluate the expansion as $x/y$ and see if $x^2 - Dy^2 = 1$.  The minimality of the solution and the termination of the algorithm are taken a bit on faith.

\footnotesize
Note: For $D = 661$ the corresponding $x = 16421658242965910275055840472270471049$.
\normalsize

\vspace{-0.5cm}
\ans{661}


%%%%%%%%%%%%%%%%%%%%%%%%%%%%%%%%%%%%%%%%%%%%%%%%%%%%%%%%%%%%%%%%%%%%%%%%%%%%%%%%

\prob{67}
By starting at the top of the triangle below and moving to adjacent numbers on the row below, the maximum total from top to bottom is 23.
\begin{center}
\begin{figure}[h]
\centering
\includegraphics[width = 0.12\textwidth]{./images/p_018.png}
\end{figure}
\end{center}
\vspace{-1cm}
That is, $3 + 7 + 4 + 9 = 23$.

Find the maximum total from top to bottom in \verb"triangle.txt", a 15k text file containing a triangle with one-hundred rows.

\footnotesize
NOTE: This is a much more difficult version of Problem 18. It is not possible to try every route to solve this problem, as there are $2^{99}$ altogether! If you could check one trillion $(10^{12})$ routes every second it would take over twenty billion years to check them all. There is an efficient algorithm to solve it. ;o)

\normalsize

\sol{67}
Well, ain't it a good thing we did problem 18 properly.  Code is identical, except a bit shorter ;-)

\ans{7273}


%%%%%%%%%%%%%%%%%%%%%%%%%%%%%%%%%%%%%%%%%%%%%%%%%%%%%%%%%%%%%%%%%%%%%%%%%%%%%%%%

\prob{68}
Consider the following ``magic'' 3-gon ring, filled with the numbers 1 to 6, and each line adding to nine.
\vspace{-0.5cm}
\begin{center}
\begin{figure}[h]
\centering
\includegraphics[width = 0.25\textwidth]{./images/p_068_1.png}
\end{figure}
\end{center}
\vspace{-1.5cm}
Working clockwise, and starting from the group of three with the numerically lowest external node (4,3,2 in this example), each solution can be described uniquely. For example, the above solution can be described by the set: 4,3,2; 6,2,1; 5,1,3.

By concatenating each group it is possible to form 9-digit strings; the maximum string for a 3-gon ring is 432621513.

Using the numbers 1 to 10, and depending on arrangements, it is possible to form 16- and 17-digit strings. What is the maximum 16-digit string for a ``magic'' 5-gon ring?

\vspace{-1cm}
\begin{center}
\begin{figure}[h]
\centering
\includegraphics[width = 0.40\textwidth]{./images/p_068_2.png}
\end{figure}
\end{center}
\vspace{-3cm}

\sol{68}
Ok, as we're told that we need a 16-digit string, that means the 10 is on the outer ring.  The maximum answer will then have
a 6, 7, 8, 9, 10 on the outside, and 1, 2, 3, 4, 5 on the inside (in some order).   In that case the total sum will be
$(1+\cdots+10)+(1+\cdots+5) = 70$, so 14 per side.  The six must get a $5+3$, the 10 must get $3+1$, the 9 then gets $1+4$,
the 8 gets $4+2$ and the 7 a $2+5$.

\ans{6531031914842725}


%%%%%%%%%%%%%%%%%%%%%%%%%%%%%%%%%%%%%%%%%%%%%%%%%%%%%%%%%%%%%%%%%%%%%%%%%%%%%%%%

\prob{69}
Euler's Totient function, $\varphi(n)$ [sometimes called the phi function], is used to determine the number of numbers less than $n$ which are relatively prime to $n$. For example, as 1, 2, 4, 5, 7, and 8, are all less than nine and relatively prime to nine, $\varphi(9)=6$.
\begin{center}
\begin{tabular}{|c|l|c|l|}
\hline
n 	& Rel. Prime 	& $\varphi(n)$ &	$n/\varphi(n)$ \\
\hline
2 	& 1 	        & 1 	& 2          \\
3 	& 1,2 	        & 2 	& 1.5        \\
4 	& 1,3 	        & 2 	& 2          \\
5 	& 1,2,3,4       & 4 	& 1.25       \\
6 	& 1,5 	        & 2 	& 3          \\
7 	& 1,2,3,4,5,6 	& 6 	& 1.1666...  \\
8 	& 1,3,5,7 	    & 4 	& 2          \\
9 	& 1,2,4,5,7,8 	& 6 	& 1.5        \\
10 	& 1,3,7,9 	    & 4 	& 2.5   \\
\hline
\end{tabular}
\end{center}
It can be seen that $n=6$ produces a maximum $n/\varphi(n)$ for $n \leq 10$.

Find the value of $n \leq 1,000,000$ for which $n/\varphi(n)$ is a maximum.

\sol{69}
It is well known that if $n = \prod_{i} p_i^{\alpha_i}$, then $\varphi(n) = \prod_i (p_i-1) \prod_i p_{i}^{\alpha_{i}-1}.$
So the ratio $n/\varphi(n) = \prod_i \left( 1-\tfrac 1{p_i}\right),$ so to maximize it, we wish to use as many small distinct primes as possible (repeating primes does not change the ratio).  So we take the largest product below the given limit:
$2\times 3 \times 5 \times 7 \times 11 \times 13 \times 17.$

\ans{510510}


%%%%%%%%%%%%%%%%%%%%%%%%%%%%%%%%%%%%%%%%%%%%%%%%%%%%%%%%%%%%%%%%%%%%%%%%%%%%%%%%

\prob{70}
Euler's Totient function, $\varphi(n)$ [sometimes called the phi function], is used to determine the number of positive numbers less than or equal to n which are relatively prime to $n$. For example, as 1, 2, 4, 5, 7, and 8, are all less than nine and relatively prime to nine, $\varphi(9)=6$.
The number 1 is considered to be relatively prime to every positive number, so $\varphi(1)=1$.

Interestingly, $\varphi(87109)=79180$, and it can be seen that 87109 is a permutation of 79180.

Find the value of $n$, $1 < n < 10^7$, for which $\varphi(n)$ is a permutation of $n$ and the ratio $n/\varphi(n)$ produces a minimum.

\sol{70}
As in the last question, $n/\varphi(n) = \prod_i \left( 1-\tfrac 1{p_i}\right)$.  Clearly, $n$ cannot be a prime, as $p$ and $p-1=\varphi(p)$ are not permutations of each other.  So we try to find the next best thing -- a product of two largish primes. So start a loop over the first prime $p_1$ from around 1000 to 1000$\sqrt{10}$, and the second prime $p_2 > p_1$, such that that the product is below $10^7$.  Check $n = p_1p_2$ and $\varphi(n) = p_1p_2 - p_1-p_2+1$ if they contain the same digits,
and check if the ratio is smaller than the previous best.  Repeat till done.  The winner is 2339$\times$3557.

\ans{8319823}

