\prob{71}
Consider the fraction, $n/d$, where $n$ and $d$ are positive integers. If $n<d$ and HCF$(n,d)=1$, it is called a reduced proper fraction.

If we list the set of reduced proper fractions for $d \leq 8$ in ascending order of size, we get:
\vspace{-0.25cm}
$$ _{1/8,\; 1/7, \; 1/6,\; 1/5,\; 1/4,\; 2/7,\; 1/3,\; 3/8,\; 2/5,\; 3/7,\; 1/2,\; 4/7,\; 3/5,\; 5/8,\; 2/3,\; 5/7,\; 3/4,\; 4/5,\; 5/6,\; 6/7,\; 7/8}$$
It can be seen that $2/5$ is the fraction immediately to the left of $3/7$.

By listing the set of reduced proper fractions for $d \leq 1,000,000$ in ascending order of size, find the numerator of the fraction immediately to the left of 3/7.

\sol{71}
Well, for a given denominator, the numerator directly to the left of it will be $N = \lfloor\tfrac{3D}{7}\rfloor$.  Running over the million denominators (except those divisible by 7), keeping track of the best one will do it.

Alternately, observe that the difference between a fraction and $3/7$ is
$$\Delta = \frac{3}{7} - \frac{N}D = \frac{3D-7N}{7D} = \frac{3D-7\lfloor\tfrac{3D}{7}\rfloor}{7D} \geq \frac1{7D},$$
so need to pick the biggest possible denominator for which $3D \equiv_7 1$, hence $D \equiv_7 5$.  This gives
$D = 10^6-3$, and $N = 428570$.

\ans{428570}


%%%%%%%%%%%%%%%%%%%%%%%%%%%%%%%%%%%%%%%%%%%%%%%%%%%%%%%%%%%%%%%%%%%%%%%%%%%%%%%%

\prob{72}
Consider the fraction, $n/d$, where $n$ and $d$ are positive integers. If $n<d$ and HCF$(n,d)=1$, it is called a reduced proper fraction.

If we list the set of reduced proper fractions for $d \leq 8$ in ascending order of size, we get:
\vspace{-0.25cm}
$$ _{1/8,\; 1/7, \; 1/6,\; 1/5,\; 1/4,\; 2/7,\; 1/3,\; 3/8,\; 2/5,\; 3/7,\; 1/2,\; 4/7,\; 3/5,\; 5/8,\; 2/3,\; 5/7,\; 3/4,\; 4/5,\; 5/6,\; 6/7,\; 7/8}$$
It can be seen that there are 21 elements in this set.

How many elements would be contained in the set of reduced proper fractions for $d \leq 1,000,000$?

\sol{72}
The number of reduced proper fractions with denominator $d$ is $\varphi(d)$, not counting $d=1$.  So we need to evaluate
$\sum_{2}^{10^6} \varphi(d)$.  Do this by recalling that $\varphi(n) = n \prod \tfrac{p_i-1}{p_i}$, so instead of looping over $n$, we loop over all the primes, and apply the factor of $\tfrac{p_i-1}{p_i}$ to every multiple of $p_i$.  And that is all that needs to be done ;-)

\ans{303963552391}


%%%%%%%%%%%%%%%%%%%%%%%%%%%%%%%%%%%%%%%%%%%%%%%%%%%%%%%%%%%%%%%%%%%%%%%%%%%%%%%%

\prob{73}
Consider the fraction, $n/d$, where $n$ and $d$ are positive integers. If $n<d$ and HCF$(n,d)=1$, it is called a reduced proper fraction.
If we list the set of reduced proper fractions for $d \leq 8$ in ascending order of size, we get:
\vspace{-0.25cm}
$$ _{1/8,\; 1/7, \; 1/6,\; 1/5,\; 1/4,\; 2/7,\; 1/3,\; 3/8,\; 2/5,\; 3/7,\; 1/2,\; 4/7,\; 3/5,\; 5/8,\; 2/3,\; 5/7,\; 3/4,\; 4/5,\; 5/6,\; 6/7,\; 7/8}$$
It can be seen that there are 3 fractions between $1/3$ and $1/2$.

How many fractions lie between $1/3$ and $1/2$ in the sorted set of reduced proper fractions for $d \leq 12,000$?

\footnotesize
Note: The upper limit has been changed recently.

\normalsize

\sol{73}
A little uninspirational.  By applying the last program to this problem, and assuming that the answer is around one sixth of the total, we'd get 7295376, so it's probably possible to brute it.  To brute it, run through the possible denominators, and the numerators will range from $\lceil \tfrac D3 \rceil $ to $\lfloor \tfrac D2\rfloor$, and the proper fraction check reduces to requiring that $(D,N) = 1$.  A simple Euclidian algorithm for the gcd will be fast enough (runs around 40s).

\ans{7295372}


%%%%%%%%%%%%%%%%%%%%%%%%%%%%%%%%%%%%%%%%%%%%%%%%%%%%%%%%%%%%%%%%%%%%%%%%%%%%%%%%

\prob{74}
The number 145 is well known for the property that the sum of the factorial of its digits is equal to 145:
$$1! + 4! + 5! = 1 + 24 + 120 = 145$$
Perhaps less well known is 169, in that it produces the longest chain of numbers that link back to 169; it turns out that there are only three such loops that exist:
\begin{eqnarray*}
&&169 \to 363601 \to 1454 \;\to \;169   \\
&&871 \to 45361  \to 871                 \\
&&872 \to 45362  \to 872
\end{eqnarray*}
It is not difficult to prove that EVERY starting number will eventually get stuck in a loop. For example,
\begin{eqnarray*}
&&69 \to 363600 \to 1454 \to 169 \to 363601 (\to 1454) \\
&&78 \to 45360 \to 871 \to 45361 (\to 871) \\
&&540\to 145 (\to 145)
\end{eqnarray*}
Starting with 69 produces a chain of five non-repeating terms, but the longest non-repeating chain with a starting number below one million is sixty terms.

How many chains, with a starting number below one million, contain exactly sixty non-repeating terms?

\vspace{-1cm}
\sol{74}
Two approaches can be taken here.  One way would be to only treat numbers which are in descending order of digits (as order 
of digits does not matter), and then look for those that take 60 steps to get a repetition.  Probably better in the long run.

Alternately, can observe that all numbers below 2.2 million map to themselves ($6\cdot 9! = 2177280)$, and do it by recursion. Keep track of the number of unique numbers collected from $n$ as $U(n)$.   Enter the known fixed values and loops, then go through the $n$ with unassigned $U(n)$.  Here $U(n) = 1+U(\sigma(n))$, with $\sigma(n)$ being the sum of the factorials of digits.  

\footnotesize
The recursion gave some trouble, so hacked it with a list-stack.  Runs in  $\sim$ 20 seconds.
\normalsize
%
\ans{402}


%%%%%%%%%%%%%%%%%%%%%%%%%%%%%%%%%%%%%%%%%%%%%%%%%%%%%%%%%%%%%%%%%%%%%%%%%%%%%%%%

\prob{75}

It turns out that 12 cm is the smallest length of wire that can be bent to form an integer sided right angle triangle in exactly one way, but there are many more examples.
\begin{eqnarray*}
\textbf{12} \text{ cm}: &&(3,4,5)    \\
\textbf{24} \text{ cm}: &&(6,8,10)   \\
\textbf{30} \text{ cm}: &&(5,12,13)  \\
\textbf{36} \text{ cm}: &&(9,12,15)  \\
\textbf{40} \text{ cm}: &&(8,15,17)  \\
\textbf{48} \text{ cm}: &&(12,16,20)
\end{eqnarray*}
In contrast, some lengths of wire, like 20 cm, cannot be bent to form an integer sided right angle triangle, and other lengths allow more than one solution to be found; for example, using 120 cm it is possible to form exactly three different integer sided right angle triangles.
$$\text{120 cm:} (30,40,50), (20,48,52), (24,45,51)$$
Given that $L$ is the length of the wire, for how many values of $L \leq 1,500,000$ can exactly one integer sided right angle triangle be formed?

\footnotesize
Note: This problem has been changed recently, please check that you are using the right parameters.
\normalsize

\sol{75}
The key to doing this problem effectively comes in generating all the primitive Pythagorean triples.  It is well known that these can all be generated by $(m^2+n^2, 2mn, m^2-n^2)$, with $(m,n) = 1$, and $m$ and $n$ having opposite parity.
Once those are known, each multiple of its perimeter is incremented by one, and in the end we look for those elements with a total of 1.


\ans{161667}


%%%%%%%%%%%%%%%%%%%%%%%%%%%%%%%%%%%%%%%%%%%%%%%%%%%%%%%%%%%%%%%%%%%%%%%%%%%%%%%%

\prob{76}
It is possible to write five as a sum in exactly six different ways:
\begin{eqnarray*}
&&4 + 1                 \\
&&3 + 2                 \\
&&3 + 1 + 1             \\
&&2 + 2 + 1             \\
&&2 + 1 + 1 + 1         \\
&&1 + 1 + 1 + 1 + 1
\end{eqnarray*}
How many different ways can one hundred be written as a sum of at least two positive integers?

\sol{76}
Well, this is an exercise in Dynamic Programming.  We keep a table $D$ for the number of ways of getting to $n$
using numbers up to $k$.  Then if can get use number $k+1$ as well, then $D(n,k+1) = D(n, k)+\sum D(n-\alpha(k+1),k)$.
Running this over the loop is very quick.  Oh, and need to subtract one to remove the `sum' of a single term.

\ans{190569291}


%%%%%%%%%%%%%%%%%%%%%%%%%%%%%%%%%%%%%%%%%%%%%%%%%%%%%%%%%%%%%%%%%%%%%%%%%%%%%%%%

\prob{77}
It is possible to write ten as the sum of primes in exactly five different ways:
\begin{eqnarray*}
&&7 + 3             \\
&&5 + 5             \\
&&5 + 3 + 2         \\
&&3 + 3 + 2 + 2     \\
&&2 + 2 + 2 + 2 + 2
\end{eqnarray*}
What is the first value which can be written as the sum of primes in over five thousand different ways?

\sol{77}
Well now, this is just a modification of the last one, where instead of using all numbers we are limited to primes.
Just need to make a list of primes, and instead of looking at $k-\alpha n$, look at $k-\alpha p_n$.  This stuff grows really quickly, and answer is below 100, surprisingly.

\ans{71}


%%%%%%%%%%%%%%%%%%%%%%%%%%%%%%%%%%%%%%%%%%%%%%%%%%%%%%%%%%%%%%%%%%%%%%%%%%%%%%%%

\prob{78}

\sol{78}

\ans{---}


%%%%%%%%%%%%%%%%%%%%%%%%%%%%%%%%%%%%%%%%%%%%%%%%%%%%%%%%%%%%%%%%%%%%%%%%%%%%%%%%

\prob{79}
A common security method used for online banking is to ask the user for three random characters from a passcode. For example, if the passcode was 531278, they may ask for the 2nd, 3rd, and 5th characters; the expected reply would be: 317.

The text file, \verb"keylog.txt", contains fifty successful login attempts.

Given that the three characters are always asked for in order, analyse the file so as to determine the shortest possible secret passcode of unknown length.

\sol{79}
Ok, so the quickest way could be with pencil and paper.  A crude bit of code which works is to make a list of which numbers
follow a given number.  Then look for which number is followed the most others -- it is probably first in line (to make this better, should update recursively), so place it first, then repeat.  The last number is never followed by anything, but it is
the number which follows the previous number.

\ans{73162890}


%%%%%%%%%%%%%%%%%%%%%%%%%%%%%%%%%%%%%%%%%%%%%%%%%%%%%%%%%%%%%%%%%%%%%%%%%%%%%%%%

\prob{80}
It is well known that if the square root of a natural number is not an integer, then it is irrational. The decimal expansion of such square roots is infinite without any repeating pattern at all.

The square root of two is 1.41421356237309504880..., and the digital sum of the first one hundred decimal digits is 475.

For the first one hundred natural numbers, find the total of the digital sums of the first one hundred decimal digits for all the irrational square roots.

\sol{80}
Ok, so the sum-of-digits functions is easy to establish.  For this we'll use Python's long integers as well a binary search.
For each value of $n$, consider $10^{99} \lfloor \sqrt n \rfloor$ and $10^{99} \lfloor (\sqrt{n}+1) \rfloor$.  They are on either side of $10^{99}\sqrt n$, which has exactly 100 digits as an integer.  A binary search with these starting values can then be run -- take $m = (n_l+n_h)/2$, and if $m^2 < 10^{198}N$, then it is the new lower limit, otherwise it is the new upper limit.  Repeat until the limits differ by 1.  The lower one gives the first 100 digits of $\sqrt n$.

\ans{40886}

