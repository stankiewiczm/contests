\prob{31}
In England the currency is made up of pound, \pounds, and pence, $p$, and there are eight coins in general circulation:
$$    1p,\; 2p,\; 5p, \;10p,\; 20p,\; 50p,\; \hbox{\pounds}1 (100p)\;\hbox{and}\;\hbox{\pounds2} (200p).$$
It is possible to make \pounds2 in the following way:
$$    1\times\hbox{\pounds1} + 1\times50p + 2\times20p + 1\times5p + 1\times2p + 3\times1p$$
How many different ways can \pounds2 be made using any number of coins?

\sol{31}
Again, we use DP.  Make a $201\times8$ array.  The first row will store the number of ways of making 
each amount from $0p$ to \pounds2 using only $1p$ coins.  The next will use two types of coins, etc.
The last (eighth) row will store the number of ways of making each amount using 8 types of coins,
and the last entry will be the answer to the problem.

\ans{73682}


%%%%%%%%%%%%%%%%%%%%%%%%%%%%%%%%%%%%%%%%%%%%%%%%%%%%%%%%%%%%%%%%%%%%%%%%%%%%%%%%

\prob{32}
We shall say that an n-digit number is pandigital if it makes use of all the digits 1 to n exactly once; for example, the 5-digit number, 15234, is 1 through 5 pandigital.

The product 7254 is unusual, as the identity, $39 \times 186 = 7254$, containing multiplicand, multiplier, and product is 1 through 9 pandigital.

Find the sum of all products whose multiplicand/multiplier/product identity can be written as a 1 through 9 pandigital.

\footnotesize
HINT: Some products can be obtained in more than one way so be sure to only include it once in your sum.
\normalsize

\sol{32}
If $a\times b = p$, notice that $p$ has to have 4 digits (if it had 3, then $a$ and $b$ would have to have 3 each, and the digits of $a$ and $b$ are at least as much as the digits of $c$).  So we can take $1 \leq a \leq 100$, and $\tfrac{1000}{a} \leq b \leq \tfrac{10000}a$.  After that it's just a case of checking all the products, being careful not to count
a product more than once. 

\ans{45228}


%%%%%%%%%%%%%%%%%%%%%%%%%%%%%%%%%%%%%%%%%%%%%%%%%%%%%%%%%%%%%%%%%%%%%%%%%%%%%%%%

\prob{33}
The fraction $\frac{49}{98}$ is a curious fraction, as an inexperienced mathematician in attempting to simplify it may incorrectly believe that $\tfrac{49}{98} = \tfrac 48$, which is correct, is obtained by cancelling the 9s.

We shall consider fractions like, $\tfrac{30}{50} = \tfrac 35$, to be trivial examples.

There are exactly four non-trivial examples of this type of fraction, less than one in value, and containing two digits in the numerator and denominator.

If the product of these four fractions is given in its lowest common terms, find the value of the denominator.

\sol{33} 
The fractions we have to consider are of the form $\tfrac{10a+b}{10c+d}$, with $1 \leq a,b,c,d \leq 9$.  Notice
 that if $a=b$, the one of $c$ or $d$ has to equal $a$ as well, and the other can be shown to be equal as well.
 So can work with $a \neq b$.  This leaves four possible
cases (which we will loop through) to work with: 
\begin{itemize}
\item[$a=c$:] Then $\tfrac{10a+b}{10a+d} = \tfrac bd\;\;\Rightarrow \;\; b=d$, but $\tfrac bd = 1$.  No solutions.
\item[$a=d$:] Then $\tfrac{10a+b}{10c+a} = \tfrac bc\;\;\Rightarrow \;\; 10ac=9bc+ab$, and also $b < c$ and $a < c$.
\item[$b=c$:] Then $\tfrac{10a+b}{10b+d} = \tfrac ad\;\; \Rightarrow\;\; 10ab=9ad+bd$, and also $a < d$ and $a < b$.
\item[$b=d$:] Then $\tfrac{10a+b}{10c+b} = \tfrac ac\;\; \Rightarrow \;\;b=0$, the trivial cases.  No solutions. 
\end{itemize}
The second case can be rewritten as $ab = ac + 9c(a-b)$.  As $b<c$, we must have $a-b=-k < 0$, for $k\in\mathbb{N}$.
But then $9kc = a(c-b)$, but $a < c$ and $c-b < 9$.  So we have a contradiction, and no solutions.

Coding up the third case gives 4 solutions:  $\tfrac{16}{64}, \tfrac{19}{95}, \tfrac{26}{65}, \tfrac{49}{98}$.
The product of those is $\tfrac 1{100}$.


\ans{100}


%%%%%%%%%%%%%%%%%%%%%%%%%%%%%%%%%%%%%%%%%%%%%%%%%%%%%%%%%%%%%%%%%%%%%%%%%%%%%%%%

\prob{34}

145 is a curious number, as 1! + 4! + 5! = 1 + 24 + 120 = 145.

Find the sum of all numbers which are equal to the sum of the factorial of their digits.

\footnotesize
Note: as 1! = 1 and 2! = 2 are not sums they are not included.
\normalsize

\sol{34}
Nothing inspirational here.  Make a look up table for the factorials from zero to nine,
run through integers up to about $10^5$ or $10^6$, checking the sum of the factorials
of the digits against the number.  Only 145 and 40585 work.

\ans{40730}


%%%%%%%%%%%%%%%%%%%%%%%%%%%%%%%%%%%%%%%%%%%%%%%%%%%%%%%%%%%%%%%%%%%%%%%%%%%%%%%%

\prob{35}
The number, 197, is called a circular prime because all rotations of the digits: 197, 971, and 719, are themselves prime.

There are thirteen such primes below 100: 2, 3, 5, 7, 11, 13, 17, 31, 37, 71, 73, 79, and 97.

How many circular primes are there below one million?

\sol{35}
Well, generate the primes up to a million, and create a binary array of primeness.  Then for each prime in the list,
move it cyclically (keeping track of the number of digits), like \verb"p0 = (p0/10) + (p0%10)*10**(Dig-1);", and check for primeness.  If all the rotations are prime, then it's good.

\ans{55}


%%%%%%%%%%%%%%%%%%%%%%%%%%%%%%%%%%%%%%%%%%%%%%%%%%%%%%%%%%%%%%%%%%%%%%%%%%%%%%%%

\prob{36}

The decimal number, $585 = 1001001001_2$ (binary), is palindromic in both bases.

Find the sum of all numbers, less than one million, which are palindromic in base 10 and base 2.

\footnotesize
(Please note that the palindromic number, in either base, may not include leading zeros.)
\normalsize

\sol{36}
Checking numbers for being palindromes is a little tricky, due to keeping track of the number of digits.
Doing it for an array is easy (just compare indices counting from both ends).  So we \verb"str" the integers in question,
and check if they're palindromes.  If so, the create a binary string of the integer and check that.  It turns out that
the easiest way to make a binary string is in reverse, as the questions asks about palindromes, that does not even matter.

\ans{872187}


%%%%%%%%%%%%%%%%%%%%%%%%%%%%%%%%%%%%%%%%%%%%%%%%%%%%%%%%%%%%%%%%%%%%%%%%%%%%%%%%

\prob{37}
The number 3797 has an interesting property. Being prime itself, it is possible to continuously remove digits from left to right, and remain prime at each stage: 3797, 797, 97, and 7. Similarly we can work from right to left: 3797, 379, 37, and 3.

Find the sum of the only eleven primes that are both truncatable from left to right and right to left.

\footnotesize
NOTE: 2, 3, 5, and 7 are not considered to be truncatable primes.
\normalsize

\sol{37} Well, this is not very difficult.  Generate a long list of primes (till a million), and create a binary array
to quickly check for primeness.  Then take each prime, and see if it passes the test, by truncating it first one way and then the other (using commands akin to \verb"p/10**k" and \verb"p%10**k").

\ans{748317}


%%%%%%%%%%%%%%%%%%%%%%%%%%%%%%%%%%%%%%%%%%%%%%%%%%%%%%%%%%%%%%%%%%%%%%%%%%%%%%%%

\prob{38}
Take the number 192 and multiply it by each of 1, 2, and 3:
\begin{eqnarray*}
    192 \times 1 &= &192 \\
    192 \times 2 &= &384 \\
    192 \times 3 &= &576
\end{eqnarray*}
By concatenating each product we get the 1 to 9 pandigital, 192384576. We will call 192384576 the concatenated product of 192 and (1,2,3)

The same can be achieved by starting with 9 and multiplying by 1, 2, 3, 4, and 5, giving the pandigital, 918273645, which is the concatenated product of 9 and (1,2,3,4,5).

What is the largest 1 to 9 pandigital 9-digit number that can be formed as the concatenated product of an integer with $(1,2, ... , n)$ where $n > 1$?

\sol{38}
Notice that the integer will form the first part of the pandigital product, so to be bigger than 918273645, it needs to begin with a 9.  Now if it has $k$ digits, the next few multiples of it will have $k+1$ digits.  So if we $n$ numbers, the concatenated product will have $n(k+1)-1 = 9$ digits.  Hence $n(k+1) = 10 = 2\cdot5$.  As $k \neq 1$, we must have $k = 4$ and $n = 2$.  Hence $9000 < k < 10000$, and need to check the digits of $k$ and $2k$ to see what works.  There are three solutions, and we take the largest one $(k,n) = (9327,2)$.

\ans{932718654}


%%%%%%%%%%%%%%%%%%%%%%%%%%%%%%%%%%%%%%%%%%%%%%%%%%%%%%%%%%%%%%%%%%%%%%%%%%%%%%%%

\prob{39}
If $p$ is the perimeter of a right angle triangle with integral length sides, $\{a,b,c\}$, there are exactly three solutions for $p = 120$.
$$ \{20,48,52\}, \quad \{24,45,51\}, \quad \{30,40,50\}$$
For which value of $p \leq 1000$, is the number of solutions maximised?

\sol{39}
Taking $a < b < c$, we have $a < 333$, and $b < 500 - \tfrac a2$.  Loop through all the possible values of $a$
and $b$, check if it gives a right-angled triangle $\sqrt{a^2+b^2} \in \mathbb{N}$, and calculate the perimeter
$p$ of each one.  Then just increment that value on the counting array, and read off the maximum.

\ans{840}


%%%%%%%%%%%%%%%%%%%%%%%%%%%%%%%%%%%%%%%%%%%%%%%%%%%%%%%%%%%%%%%%%%%%%%%%%%%%%%%%

\prob{40}
An irrational decimal fraction is created by concatenating the positive integers:
$$0.12345678910\textbf{1}112131415161718192021...$$
It can be seen that the $12^{th}$ digit of the fractional part is 1.

If $d_n$ represents the $n^{th}$ digit of the fractional part, find the value of the following expression.
$$d_1 \times d_{10} \times d_{100} \times d_{1000} \times d_{10000} \times d_{100000} \times d_{1000000}$$

\sol{40}
There are better ways of doing it, but as the last digit we need to find the millionth one, it can just be done
iteratively.  Keep track of the current number in the sequence, the number of digits used up, and the next upcoming
power of 10.  In normal circumstances, we just update the number and digits, but if we cross a power of 10, there
is a need to pull out the relevant digit.  As we'd need the $(10^k-d)$'th digit of a number, it's not hard to pull it out
with either modulo arithmetic, or string manipulation.

\ans{210}


%%%%%%%%%%%%%%%%%%%%%%%%%%%%%%%%%%%%%%%%%%%%%%%%%%%%%%%%%%%%%%%%%%%%%%%%%%%%%%%%

