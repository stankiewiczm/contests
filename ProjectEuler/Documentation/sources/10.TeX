\prob{1}

If we list all the natural numbers below 10 that are multiples of 3 or 5, we get 3, 5, 6 and 9. The sum of these multiples is 23.

Find the sum of all the multiples of 3 or 5 below 1000.

\sol{1}

While an analytical solution can be found, it's simplest to just run a loop over all numbers up to 1000
and sum the ones that satisfy the criteria.

\ans{233168}


%%%%%%%%%%%%%%%%%%%%%%%%%%%%%%%%%%%%%%%%%%%%%%%%%%%%%%%%%%%%%%%%%%%%%%%%%%%%%%%%


\prob{2}
Each new term in the Fibonacci sequence is generated by adding the previous two terms. By starting with 1 and 2, the first 10 terms will be:
$$1, 2, 3, 5, 8, 13, 21, 34, 55, 89, ...$$
Find the sum of all the even-valued terms in the sequence which do not exceed four million.

\sol{2}

The easiest way to do this problem is to generate the Fibonacci sequence directly.  Start by creating a list with the first
two terms as \verb"[1,1]",
and generate the rest with \verb"FIB.append(FIB[-1]+FIB[-2]);".  Finding the sum of the even terms is trivial.

\ans{4613732}


%%%%%%%%%%%%%%%%%%%%%%%%%%%%%%%%%%%%%%%%%%%%%%%%%%%%%%%%%%%%%%%%%%%%%%%%%%%%%%%%

\prob{3}

The prime factors of 13195 are 5, 7, 13 and 29.

What is the largest prime factor of the number 600851475143 ?

\sol{3}

The number given is odd, so all the prime factors will be odd.  We start by assigning the number to $N$,
and then progress through all the odd numbers.  If $N$ is divisible by a number, we take out the factor and
repeat, otherwise we move onto the next odd number.  Repeat until $N = 1$, and the last factor found is the answer.

\ans{6857}


%%%%%%%%%%%%%%%%%%%%%%%%%%%%%%%%%%%%%%%%%%%%%%%%%%%%%%%%%%%%%%%%%%%%%%%%%%%%%%%%


\prob{4}

A palindromic number reads the same both ways. The largest palindrome made from the product of two 2-digit numbers is $9009 = 91 \times 99$.

Find the largest palindrome made from the product of two 3-digit numbers.

\sol{4}

Checking if a number is a 6-digit palindrome is the only messy part here.  A product of two 3-digit numbers will
be less than $10^6$, so need to first check that it is greater than $10^5$.  Then check for equality between first and last digits,
the second and fifth digits, and finally the third and fourth digits.  A function to return \verb"True" or \verb"False" works best.

Then it's just a question of looping over all 3-digit numbers $n_1$ and $n_2$ and checking the products, keeping track of the current maximum.

\ans{906609}


%%%%%%%%%%%%%%%%%%%%%%%%%%%%%%%%%%%%%%%%%%%%%%%%%%%%%%%%%%%%%%%%%%%%%%%%%%%%%%%%


\prob{5}

2520 is the smallest number that can be divided by each of the numbers from 1 to 10 without any remainder.

What is the smallest number that is evenly divisible by all of the numbers from 1 to 20?

\sol{5}

Some numbers have factors in common.  The least common multiple needs to be divisible by every prime only
as many times as the highest power of it in the list of numbers involved.  So by checking only the powers of primes,
$$ LCM = 16\times9\times7\times11\times13\times17\times19.$$

\ans{232792560}


%%%%%%%%%%%%%%%%%%%%%%%%%%%%%%%%%%%%%%%%%%%%%%%%%%%%%%%%%%%%%%%%%%%%%%%%%%%%%%%%

\prob{6}

The sum of the squares of the first ten natural numbers is,
$$1^2 + 2^2 + ... + 10^2 = 385$$
The square of the sum of the first ten natural numbers is,
$$(1 + 2 + ... + 10)^2 = 55^2 = 3025$$

Hence the difference between the sum of the squares of the first ten natural numbers and the square of the sum is $3025 - 385 = 2640$.

Find the difference between the sum of the squares of the first one hundred natural numbers and the square of the sum.

\sol{6} Recall that
$$\sum_{k=1}^n k = \frac{n(n+1)}2\qquad \hbox{ and } \qquad \sum_{k=1}^n k^2 = \frac{n(n+1)(2n+1)}{6}.$$
The answer is simply
$$\frac{n^2(n+1)^2}4 - \frac{n(n+1)(2n+1)}{6} \qquad \hbox{ for }n= 100.$$

\ans{25164150}


%%%%%%%%%%%%%%%%%%%%%%%%%%%%%%%%%%%%%%%%%%%%%%%%%%%%%%%%%%%%%%%%%%%%%%%%%%%%%%%%

\prob{7}
By listing the first six prime numbers: 2, 3, 5, 7, 11, and 13, we can see that the 6$^{\textrm{th}}$ prime is 13.

What is the 10001$^{\textrm{st}}$ prime number?

\sol{7}

The general implementation here is that of the ``Sieve of Eratosthenes''.  An array of some size is created,
with all entries (except 0 and 1) set to \verb"1" (for prime, a \verb"0" means not-prime).
We proceed by finding the smallest number labeled with \verb"1", and mark all it's multiples as non-prime.
In addition, store this number in a list of primes.  Repeat until the end.  Then read off the relevant prime.

\ans{104743}


%%%%%%%%%%%%%%%%%%%%%%%%%%%%%%%%%%%%%%%%%%%%%%%%%%%%%%%%%%%%%%%%%%%%%%%%%%%%%%%%

\prob{8}
Find the greatest product of five consecutive digits in the 1000-digit number.
\begin{center}
\verb"73167176531330624919225119674426574742355349194934"
\verb"96983520312774506326239578318016984801869478851843"
\verb"85861560789112949495459501737958331952853208805511"
\verb"12540698747158523863050715693290963295227443043557"
\verb"66896648950445244523161731856403098711121722383113"
\verb"62229893423380308135336276614282806444486645238749"
\verb"30358907296290491560440772390713810515859307960866"
\verb"70172427121883998797908792274921901699720888093776"
\verb"65727333001053367881220235421809751254540594752243"
\verb"52584907711670556013604839586446706324415722155397"
\verb"53697817977846174064955149290862569321978468622482"
\verb"83972241375657056057490261407972968652414535100474"
\verb"82166370484403199890008895243450658541227588666881"
\verb"16427171479924442928230863465674813919123162824586"
\verb"17866458359124566529476545682848912883142607690042"
\verb"24219022671055626321111109370544217506941658960408"
\verb"07198403850962455444362981230987879927244284909188"
\verb"84580156166097919133875499200524063689912560717606"
\verb"05886116467109405077541002256983155200055935729725"
\verb"71636269561882670428252483600823257530420752963450"
\end{center}

\sol{8}
The highest possible product is $9^5$, then $8\cdot9^4$, then $8^2\cdot 9^3$, then $7\cdot8\cdot9^3$.
By inspection, the first three are not present, and the fourth one is, so that is the answer.

\ans{40824}


%%%%%%%%%%%%%%%%%%%%%%%%%%%%%%%%%%%%%%%%%%%%%%%%%%%%%%%%%%%%%%%%%%%%%%%%%%%%%%%%

\prob{9} A Pythagorean triplet is a set of three natural numbers, $a < b < c$, for which,
$$a^2 + b^2 = c^2$$
For example, $3^2 + 4^2 = 9 + 16 = 25 = 5^2$.

There exists exactly one Pythagorean triplet for which $a + b + c = 1000$.
Find the product abc.

\sol{9} As $(a,b,c)$ is a Pythagorean triplet, we know that $a < 300$, $b < 500$, so we can loop
over all the possible values of $a$ and $b$, solve for $c$ and check the sum.

Alternately, one can notice the primitive triplet $(8,15,17)$ has a perimeter of 40, which is a factor of
1000, so the triplet $25\times(8,15,17) = (200, 375, 425)$ has to be the one we need.

\ans{31875000}


%%%%%%%%%%%%%%%%%%%%%%%%%%%%%%%%%%%%%%%%%%%%%%%%%%%%%%%%%%%%%%%%%%%%%%%%%%%%%%%%

\prob{10}
The sum of the primes below 10 is 2 + 3 + 5 + 7 = 17.

Find the sum of all the primes below two million.

\sol{10}
As in problem 7, we use the sieve of Eratosthenes.  This time we know the upper bound (two million),
so we just find all the primes, and find their sum

\ans{142913828922}

