\prob{121}
A bag contains one red disc and one blue disc. In a game of chance a player takes a disc at random and its colour is noted. After each turn the disc is returned to the bag, an extra red disc is added, and another disc is taken at random.

The player pays \pounds1 to play and wins if they have taken more blue discs than red discs at the end of the game.

If the game is played for four turns, the probability of a player winning is exactly $11/120$, and so the maximum prize fund the banker should allocate for winning in this game would be \pounds 10 before they would expect to incur a loss. Note that any payout will be a whole number of pounds and also includes the original \pounds 1 paid to play the game, so in the example given the player actually wins \pounds9.

Find the maximum prize fund that should be allocated to a single game in which fifteen turns are played.

\sol{121}
It would seem tempting to run a simulation of the game, but is actually fairly easy to do it `properly'.  The chance of drawing a blue ball in draw $k$ is $\tfrac{1}{k+1}$.  For a 15 ball game, there are $2^{15} = 32768$ ways of setting up a blue-red pattern, and each one has an appropriate probability (as we know the probability of each colour at each draw).  The sum of these can be checked to come to zero, but we only sum those with 8 or more blue balls.  The array operations are useful in this question.  Once we have the probability of success $P$, the maximal payout is $\lfloor \tfrac 1P \rfloor$.

\ans{2269}


%%%%%%%%%%%%%%%%%%%%%%%%%%%%%%%%%%%%%%%%%%%%%%%%%%%%%%%%%%%%%%%%%%%%%%%%%%%%%%%%

\prob{122}
The most naive way of computing $n^{15}$ requires fourteen multiplications:
$$n \times n � ... \times n = n^{15}$$
But using a ``binary'' method you can compute it in six multiplications:
\begin{eqnarray*}
n   \times n & = &n^2      \\
n^2 \times n^2 & = &n^4    \\
n^4 \times n^4 & = &n^8    \\
n^8 \times n^4 & = &n^12   \\
n^{12} \times n^2 & = &n^{14}  \\
n^{14} \times n & = &n^{15}    
\end{eqnarray*}
However it is yet possible to compute it in only five multiplications:
\begin{eqnarray*}
n \times n & = & n^2      \\
n^2 \times n & = & n^3    \\
n^3 \times n^3 & = & n^6  \\
n^6 \times n^6 & = & n^{12} \\
n^{12} \times n^3 & = & n^{15}
\end{eqnarray*}
We shall define $m(k)$ to be the minimum number of multiplications to compute $n^k$; for example $m(15) = 5$.

For $1 \leq k \leq 200$, find $\sum m(k)$.

\sol{122}
The idea is to find the optimal solution for each $n$, keeping track of which powers were found in order to get there, then 
going up in $n$, checking for which smaller $n_0$ one can obtain $n$ in one more multiplication.  The messy part is that one ought to keep $all$ the best combinations, otherwise some numbers might not get an optimal solution (as happened for $n=77$).
It can be noted that $m(2n) = m(n)+1$ for all $n$, and that makes the calculations a bit quicker.
 
\ans{1582}


%%%%%%%%%%%%%%%%%%%%%%%%%%%%%%%%%%%%%%%%%%%%%%%%%%%%%%%%%%%%%%%%%%%%%%%%%%%%%%%%

\prob{123}
Let $p_n$ be the $n^\text{th}$ prime: 2, 3, 5, 7, 11, ..., and let $r$ be the remainder when
$(p_n-1)^n + (p_n+1)^n$ is divided by $p_n^2$.

For example, when $n = 3$, $p_3 = 5$, and $4^3 + 6^3 = 280 \equiv 5$ mod 25.

The least value of $n$ for which the remainder first exceeds $10^9$ is 7037.

Find the least value of n for which the remainder first exceeds $10^{10}$.

\sol{123}
There are two cases to deal with here.  If $n$ is even, then
$$(p_n-1)^n + (p_n+1)^n \equiv_{p_n^2} np_n (-1)^{n-1} + (-1)^n + np_n \; 1^{n-1} + 1^n \equiv_{p_n^2} 2.$$
In the case where $n$ is odd however,
$$(p_n-1)^n + (p_n+1)^n \equiv_{p_n^2} np_n (-1)^{n-1} + (-1)^n + np_n \; 1^{n-1} + 1^n \equiv_{p_n^2} 2np_n.$$
So we just need to run through odd $n$, checking if $2np_n > 10^{10}$.  This occurs for
$(n, p_n) = (21035, 237737)$.

\ans{21035}


%%%%%%%%%%%%%%%%%%%%%%%%%%%%%%%%%%%%%%%%%%%%%%%%%%%%%%%%%%%%%%%%%%%%%%%%%%%%%%%%

\prob{124}
The radical of $n$, rad$(n)$, is the product of distinct prime factors of $n$. For example, $504 = 2^3 \times 3^2 \times 7$, so rad$(504) = 2 \times 3 \times 7 = 42$.

If we calculate rad$(n)$ for $1 \leq n \leq 10$, then sort them on rad($n$), and sorting on $n$ if the radical values are equal, we get:
\begin{center}
\begin{tabular}{ccp{2cm}ccc}
%Unsorted &&& Sorted && \\
    $n$ & rad($n$) & & $n$ & rad($n$) & $k$ \\
    1 & 1 & & 1 & 1 & 1 \\
    2 & 1 & & 2 & 2 & 2 \\
    3 & 3 & & 4 & 2 & 3 \\
    4 & 2 & & 8 & 2 & 4 \\
    5 & 4 & & 3 & 3 & 5 \\
    6 & 6 & & 9 & 3 & 6 \\
    7 & 7 & & 5 & 5 & 7 \\
    8 & 2 & & 6 & 6 & 8 \\
    9 & 3 & & 7 & 7 & 9 \\
   10 &10 & &10 & 10 & 10
\end{tabular}
\end{center}

Let $E(k)$ be the $k^\text{th}$ element in the sorted n column; for example, $E(4) = 8$ and $E(6) = 9$.

If rad($n$) is sorted for $1 \leq n \leq 100000$, find $E(10000)$.

\sol{124}
Well, we need to calculate rad$(n)$ for each $n$ in the range, and as it consists of the product of the constituent primes, we can generated this table while looking for the primes (modified Sieve algorithm).  We can then sort the radical values to find the 10000$^\text{th}$ one.  However, there will be several numbers that map to that radical.  To fix this, I chose to encode the number into the `radical', by storing rad$(n)+\tfrac{n}{10^6}$.  The sort works just as well, and we can then reconstruct the original number, as is required.

\ans{21417}


%%%%%%%%%%%%%%%%%%%%%%%%%%%%%%%%%%%%%%%%%%%%%%%%%%%%%%%%%%%%%%%%%%%%%%%%%%%%%%%%

\prob{125}
The palindromic number 595 is interesting because it can be written as the sum of consecutive squares:
$6^2 + 7^2 + 8^2 + 9^2 + 10^2 + 11^2 + 12^2$.

There are exactly eleven palindromes below one-thousand that can be written as consecutive square sums, and the sum of these palindromes is 4164. Note that $1 = 0^2 + 1^2$ has not been included as this problem is concerned with the squares of positive integers.

Find the sum of all the numbers less than $10^8$ that are both palindromic and can be written as the sum of consecutive squares.

\sol{125}
While the problem seems daunting at first, there is a great simplification we can do.  First observe that since a sum of squares includes at least two terms, the largest square to consider is $7071^2$.  Next, let $S_k = \sum_1^k i^2$.  Then any sum of consecutive squares is a a difference of two $S_k$'s.  And while there are 25 million such terms, less than 400000 of them are below $10^8$.  Those just need to be checked if they are palindromes, and added to the list (need to watch for repeats, of which there are two).

\ans{2906969179}


%%%%%%%%%%%%%%%%%%%%%%%%%%%%%%%%%%%%%%%%%%%%%%%%%%%%%%%%%%%%%%%%%%%%%%%%%%%%%%%%

\prob{126}

\sol{126}

\ans{---}


%%%%%%%%%%%%%%%%%%%%%%%%%%%%%%%%%%%%%%%%%%%%%%%%%%%%%%%%%%%%%%%%%%%%%%%%%%%%%%%%

\prob{127}

\sol{127}

\ans{---}


%%%%%%%%%%%%%%%%%%%%%%%%%%%%%%%%%%%%%%%%%%%%%%%%%%%%%%%%%%%%%%%%%%%%%%%%%%%%%%%%

\prob{128}

\sol{128}

\ans{---}


%%%%%%%%%%%%%%%%%%%%%%%%%%%%%%%%%%%%%%%%%%%%%%%%%%%%%%%%%%%%%%%%%%%%%%%%%%%%%%%%

\prob{129}
A number consisting entirely of ones is called a repunit. We shall define $R(k)$ to be a repunit of length $k$; for example, $R(6) = 111111$.

Given that $n$ is a positive integer and GCD$(n, 10) = 1$, it can be shown that there always exists a value, $k$, for which $R(k)$ is divisible by $n$, and let $A(n)$ be the least such value of $k$; for example, $A(7) = 6$ and $A(41) = 5$.

The least value of $n$ for which $A(n)$ first exceeds ten is 17.

Find the least value of $n$ for which $A(n)$ first exceeds one-million.

\sol{129}
Well, since $R(k) = \tfrac 19(10^k-1)$, observe that if $k|10^n-1$ then $A(n) \leq k$.  And since, by Euler's theorem,
$10^{\varphi(n)} \equiv_n 1$, we know that $A(n) | \varphi(n)$.  To find the smallest $n$ for which $A(n)$ is greater than a million, it should be easiest to start looking for above a million, and evaluating their $A(n)$ directly.

For numbers not divisible by three, can just check when $10^k \equiv_n 1$.  For multiples of three, can evaluate the repunit
(mod p) directly.  Slow, but gets the job done.

\ans{1000023}

%%%%%%%%%%%%%%%%%%%%%%%%%%%%%%%%%%%%%%%%%%%%%%%%%%%%%%%%%%%%%%%%%%%%%%%%%%%%%%%%

\prob{130}
A number consisting entirely of ones is called a repunit. We shall define R(k) to be a repunit of length k; for example,
$R(6) = 111111$.

Given that $n$ is a positive integer and GCD$(n, 10) = 1$, it can be shown that there always exists a value, $k$, for which $R(k)$ is divisible by $n$, and let $A(n)$ be the least such value of $k$; for example, $A(7) = 6$ and $A(41) = 5$.

You are given that for all primes, $p > 5$, that $p - 1$ is divisible by $A(p)$. For example, when $p = 41$, $A(41) = 5$, and 40 is divisible by 5.

However, there are rare composite values for which this is also true; the first five examples being 91, 259, 451, 481, and 703.

Find the sum of the first 25 composite values of $n$ for which
GCD$(n, 10) = 1$ and $n - 1$ is divisible by $A(n)$.

\sol{130}
If $R(A(n))$ is divisible by $n$, and $n-1$ is divisible by $A(n)$, then $R(n-1)$ willa also divisible by $n$.  And as
$R(n) = \tfrac 19(10^n-1)$, for $n$ not divisible by 3, we need to check if $10^n\equiv_n 1$.  We use the \verb"pow" function,
and the code runs nice and fast.

\ans{149253}

