\prob{41}
We shall say that an $n$-digit number is pandigital if it makes use of all the digits 1 to $n$ exactly once. For example, 2143 is a 4-digit pandigital and is also prime.

What is the largest $n$-digit pandigital prime that exists?

\sol{41}
For a start $n \leq 9$.  But if $n=9$, the sum of the digits is 45, and if $n=8$ the sum of
the digits is 36.  In each case the number will be divisible by 3.  So we will consider $n=7$.
The largest 7-digit pandigital is 7654321, so generate primes up to that limit.  Then
simply read down the list of primes until one of them is 7-digit pandigital.

\ans{7652413}


%%%%%%%%%%%%%%%%%%%%%%%%%%%%%%%%%%%%%%%%%%%%%%%%%%%%%%%%%%%%%%%%%%%%%%%%%%%%%%%%

\prob{42}
The $n^{th}$ term of the sequence of triangle numbers is given by, $t_n = \tfrac 12n(n+1)$; so the first ten triangle numbers are:
$$ 1, 3, 6, 10, 15, 21, 28, 36, 45, 55, ... $$
By converting each letter in a word to a number corresponding to its alphabetical position and adding these values we form a word value. For example, the word value for SKY is 19 + 11 + 25 = 55 = $t_{10}$. If the word value is a triangle number then we shall call the word a triangle word.

Using \verb"words.txt" (right click and 'Save Link/Target As...'), a 16K text file containing nearly two-thousand common English words, how many are triangle words?

\sol{42}
Well, read in the words, and work out the total value of each word.  To check whether it is a triangle number, one could check properly, or one can just make a list of the first 20 triangular numbers and to check if the number is in the list...


\ans{162}


%%%%%%%%%%%%%%%%%%%%%%%%%%%%%%%%%%%%%%%%%%%%%%%%%%%%%%%%%%%%%%%%%%%%%%%%%%%%%%%%

\prob{43}
The number, 1406357289, is a 0 to 9 pandigital number because it is made up of each of the digits 0 to 9 in some order, but it also has a rather interesting sub-string divisibility property.

Let $d_1$ be the $1^{st}$ digit, $d_2$ be the $2^{nd}$ digit, and so on. In this way, we note the following:
\begin{itemize}                             \vspace{-0.3cm}
\item $d_2d_3d_4=406$ is divisible by 2     \vspace{-0.3cm}
\item $d_3d_4d_5=063$ is divisible by 3     \vspace{-0.3cm}
\item $d_4d_5d_6=635$ is divisible by 5     \vspace{-0.3cm}
\item $d_5d_6d_7=357$ is divisible by 7     \vspace{-0.3cm}
\item $d_6d_7d_8=572$ is divisible by 11    \vspace{-0.3cm}
\item $d_7d_8d_9=728$ is divisible by 13    \vspace{-0.3cm}
\item $d_8d_9d_{10}=289$ is divisible by 17 \vspace{-0.3cm}
\end{itemize}                   

Find the sum of all 0 to 9 pandigital numbers with this property.

\sol{43}
Firstly notice that $d_6$ is either 0 or 5.  If it is zero, then $d_7d_8$ is divisible by 11, so $d_7 = d_8$, which
is impossible.  So $d_6 = 5$, and there are 8 possibilities for $(d_7, d_8)$, namely:
(6,1), (7,2), (8,3), (9,4), (0,6), (1,7), (2,8) and (3,9).   From this, 
the first, fourth, fifth and sixth allow no solutions once we require that $d_7d_8d_9 \equiv_{13} 0$ (they 
have either no solution for $d_9$, or they reuse a digit).  \\
Looking at $d_{10}$, we have  one more elimination, (\#3), as  requires $d_8 = 3 = d_{10}$. \\
Looking back at $d_4$, one more case is eliminated (\#8), as  requires $d_4 = 5 = d_5$. \\
So the last six digits are of the possible two forms:
$$ 357289 \qquad \hbox{ or } \qquad 952867.$$
Now both $d_5 \equiv_3 $, hence must have that $d_3+d_4 \equiv_3 0$, and $d_4$ must be even.  
In both cases, the digits 2,5 and 8 are already used, so $d_4 \neq 4$.  So in the first case:
$(d_3, d_4) = $ (6,0) or (0,6), and in the second case $(d_3,d_4) = (3,0)$.  In both cases, 
the first two digits have to be 1 and 4 (in either order).  So there are 6 solutions are: 
$$ 1406357289, \qquad 1460357289, \qquad 1430952867,$$      \vspace{-0.5cm}
$$ 4106357289, \qquad 4160357289, \qquad 4130952867.$$

\vspace{-0.5cm}
\ans{16695334890}


%%%%%%%%%%%%%%%%%%%%%%%%%%%%%%%%%%%%%%%%%%%%%%%%%%%%%%%%%%%%%%%%%%%%%%%%%%%%%%%%

\prob{44}
Pentagonal numbers are generated by the formula, $P_n=n(3n-1)/2$. The first ten pentagonal numbers are:
$$ 1, 5, 12, 22, 35, 51, 70, 92, 117, 145, ...$$
It can be seen that $P_4 + P_7 = 22 + 70 = 92 = P_8$. However, their difference, 70 - 22 = 48, is not pentagonal.

Find the pair of pentagonal numbers, $P_j$ and $P_k$, for which their sum and difference is pentagonal and $D = |P_k - P_j|$ is minimised; what is the value of $D$?

\sol{44}
We're told that $P_n = \tfrac{n(3n-1)}2$, so $6P_n = 9n^2-3n = (3n-\tfrac 12)^2-\tfrac 14$.  This gives
us a quick check if a number is pentagonal or not -- just consider $\tfrac 12 + \sqrt{6P_n+\tfrac 14}$ and
check if it's an integer divisible by 3.

So now we just write a criminally inefficient solution by looping from 1 to 2500 over each of a pair
of numbers $n_1$ and $n_2$, checking for whether $P(n_1)-P(n_2)$ is pentagonal, and then for $P(n_1)+P(n_2)$.
The first solution is $(2167, 1020)$.

\ans{5482660}


%%%%%%%%%%%%%%%%%%%%%%%%%%%%%%%%%%%%%%%%%%%%%%%%%%%%%%%%%%%%%%%%%%%%%%%%%%%%%%%%

\prob{45}
Triangle, pentagonal, and hexagonal numbers are generated by the following formulae:  \\
Triangle\phantom{aln }    $\quad T_n=n(n+1)/2   \qquad\;    1, 3, 6, 10, 15, ...         $   \\
Pentagonal 	$\quad   P_n=n(3n-1)/2        \qquad	  	1, 5, 12, 22, 35, ...   $     \\
Hexagonal\phantom{ } 	$\quad   H_n=n(2n-1) 	  \qquad\;\;\; 	1, 6, 15, 28, 45, ...   $        \\

It can be verified that $T_{285} = P_{165} = H_{143} = 40755$.

Find the next triangle number that is also pentagonal and hexagonal.

\sol{45}
It is easy to see that $H_k = T_{2k-1}$, so we simply need to find the next number which
is both pentagonal and hexagonal.  For problem 44 we used a formula to check
for pentagonal-ness, so just run through hexagonal numbers, use it again looking for a match.
It turns out to be $T_{55385} = P_{31977} = H_{27693}$

\ans{1533776805}


%%%%%%%%%%%%%%%%%%%%%%%%%%%%%%%%%%%%%%%%%%%%%%%%%%%%%%%%%%%%%%%%%%%%%%%%%%%%%%%%

\prob{46}

It was proposed by Christian Goldbach that every odd composite number can be written as the sum of a prime and twice a square.
\begin{eqnarray*}
9 &= &7 + 2\times1^2\\
15 &= &7 + 2\times2^2\\
21 &= &3 + 2\times3^2\\
25 &= &7 + 2\times3^2\\
27 &= &19 + 2\times2^2\\
33 &= &31 + 2\times1^2
\end{eqnarray*}
It turns out that the conjecture was false.

What is the smallest odd composite that cannot be written as the sum of a prime and twice a square?

\sol{46}
Generate all the primes up to $10^4$.  Make a sufficiently large boolean array and mark off
all those numbers that can be written as $p + 2n^2$, for $n \geq 100$.  Then go through the odd
numbers until you find one which is not a prime and not do-able.

\ans{5777}


%%%%%%%%%%%%%%%%%%%%%%%%%%%%%%%%%%%%%%%%%%%%%%%%%%%%%%%%%%%%%%%%%%%%%%%%%%%%%%%%

\prob{47}
The first two consecutive numbers to have two distinct prime factors are:
\begin{eqnarray*}
14 &= &2 \times 7 \\
15 &= &3 \times 5
\end{eqnarray*}
The first three consecutive numbers to have three distinct prime factors are:
\begin{eqnarray*}
644 &= &2^2 \times 7 \times 23  \\
645 &= &3 \times 5 \times 43    \\
646 &= &2 \times 17 \times 19.
\end{eqnarray*}
Find the first four consecutive integers to have four distinct primes factors. What is the first of these numbers?

\sol{47}
Well, start by generating all the primes up to $10^6$.  Then create an array of zeros up to that limit.
For each prime $p$, add one to the array for each element which is a multiple of $p$.  The array will
thus store the number of distinct prime factors of each number.  Then it's just a case of finding
four consecutive elements equal to 4.

\ans{134043}


%%%%%%%%%%%%%%%%%%%%%%%%%%%%%%%%%%%%%%%%%%%%%%%%%%%%%%%%%%%%%%%%%%%%%%%%%%%%%%%%

\prob{48}
The series, $1^1 + 2^2 + 3^3 + ... + 10^{10} = 10405071317.$

Find the last ten digits of the series, $1^1 + 2^2 + 3^3 + ... + 1000^{1000}.$

\sol{48}
If ever there was a time to abuse the \verb"Python" long integers this is it.  Just evaluate the entire sum
and print out the last 10 digits...

\ans{9110846700}


%%%%%%%%%%%%%%%%%%%%%%%%%%%%%%%%%%%%%%%%%%%%%%%%%%%%%%%%%%%%%%%%%%%%%%%%%%%%%%%%

\prob{49}
The arithmetic sequence, 1487, 4817, 8147, in which each of the terms increases by 3330, is unusual in two ways: (i) each of the three terms are prime, and, (ii) each of the 4-digit numbers are permutations of one another.

There are no arithmetic sequences made up of three 1-, 2-, or 3-digit primes, exhibiting this property, but there is one other 4-digit increasing sequence.

What 12-digit number do you form by concatenating the three terms in this sequence?

\sol{49}
This question takes a little finesse.  Start by generating all the 4-digit prime numbers.  Then find all pairs
of primes for which the average is also a prime -- this is a fairly large number.  We then write a function which will
check for likeness of digits: Comparing three numbers is a little tricky, so do it in pairs.  Make an array of size 10,
and for each digit of $p_1$ increment the counter corresponding to the digit.  Then decrement the relevant counter
for each digit of $p_2$.  The array contains only zeros iff the numbers are permutations of each other.  Do it again for
$p_1$ and $p_3$ to finish the job.  The second triplet is (2969, 6299, 9629).

\footnotesize
As it turns out, the common difference is again 3330, and the problem could have been solved a lot quicker
had we assumed that to be the case.
\normalsize

\ans{296962999629}


%%%%%%%%%%%%%%%%%%%%%%%%%%%%%%%%%%%%%%%%%%%%%%%%%%%%%%%%%%%%%%%%%%%%%%%%%%%%%%%%

\prob{50}
The prime 41, can be written as the sum of six consecutive primes:
$$41 = 2 + 3 + 5 + 7 + 11 + 13$$
This is the longest sum of consecutive primes that adds to a prime below one-hundred.

The longest sum of consecutive primes below one-thousand that adds to a prime, contains 21 terms, and is equal to 953.

Which prime, below one-million, can be written as the sum of the most consecutive primes?

\sol{50}
Start by generating the primes up to a million (for a boolean look-up table).  Then notice that the sum of the 
first 547 primes is above a million, so our sum will consist of less terms.  Create a \verb"S" array, such 
that the element \verb"S[k]" will contain the sum of the first \verb"k" terms.  A sum of primes will be a difference
of two of these sums.  We start at the top, looking at elements 546 terms apart, and checking the difference for
primeness.  Then decrease the number of elements until an answer is found.  It turns out to be
for 543 terms, leaving out the first 3...

\ans{997651}


%%%%%%%%%%%%%%%%%%%%%%%%%%%%%%%%%%%%%%%%%%%%%%%%%%%%%%%%%%%%%%%%%%%%%%%%%%%%%%%%

